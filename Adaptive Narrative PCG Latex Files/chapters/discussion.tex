\section{Discussion} \label{Discussion}
Given the preliminary results from \ref{Results} as well as our own experience with the project, we believe there are some issues to address with our current implementation. We found that it currently takes a lot of writing and tagging to keep the story consistent and that the system is difficult to debug due to the random nature of the model. Minor mistakes with tagging lead to undefined behavior or crashes. Our system may be procedural in design, but it still requires a developer to put a significant amount of time into writing characters and potential storylines. We did not have enough time to reach the level of detail we hoped in our demo, and believe additional writing/debugging is needed to showcase the strengths of our model.

The random nature of the system could lead to repetitive behavior. We designed a fight scene that could last a variable number of events, and many users became convinced they were stuck in a loop and tried the "run away" option. The obvious solution to this would be to prevent specific events from occurring again, however, doing this eliminates generic events that can be used in a number of different places. For example, it would make the developer have to write a different death scene for every possible character; this would effectively reduce our model to a random decision tree. Discussion on how to resolve this narrative problem is ongoing.

Renpy was an easy platform to test our ideas on, but many of the engine's features are not adapted for the random nature of PCG. The game is designed to support a GOTO style of event transition which is not supportive of randomly and dynamically changing the story during runtime. Due to this, core Renpy features such as saving and the back button do not function as intended. Renpy provides an easy visual framework to test our ideas on, but further testing would need to be done to see if the engine can fully support PCG.
