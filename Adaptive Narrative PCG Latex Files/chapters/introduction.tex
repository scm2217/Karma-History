\section{Introduction} \label{Introduction}
Story is the backbone of many games. Traditionally, writers must spend a large amount of time putting a story together that makes the player feel like their decisions are impacting the game. This paper presents an automated approach that aims to reduce the amount of thought needed for new storylines by generating them on the fly and adapting them to a user's actions.

Currently, content generation through grammars is a solution that is applied to this problem. Grammars allow many potential options to be used at a given point. However, they have the major downside of being too static. They are unable to handle to handle variables during runtime, meaning they will not be able to act on history or provide unique options to different player types.

The most common approach in the modern game industry is the use of decisions trees. This approach is essentially hardcoded and will always have a set number of outcomes in the end. Using decision trees always requires a developer to manually write every potential story event, many of which are never seen by the majority of players. The cost to return ratio of this method forces developers to significantly limit either the depth or amount of storylines that can appear in their narrative.

\subsection{Our Contribution}
We aim to show, with a prototype in Renpy, that our text-based adventure generation model can create complex stories with many branches. Story events are decided on the fly as new events open and close based on what the player does. They can be written modularly and inserted, reducing the need for knowledge of the whole story, and making the addition of new events easier. Our player history system allows story events to be influenced by player personality and can be used to determine what options are available to a player at any given point.

\subsection{Paper Outline}
In the remainder of this paper, Section \ref{Background} covers previous works to narrative PCG (procedural content generation), Section \ref{Methods} covers how the prototype works at a high level, Section \ref{Results} shows user surveys results on the prototype, Section \ref{Discussion} analyzes these survey results to determine the viability of the prototype, and Section \ref{Conclusion} finishes with final thoughts and future works.

\section{Background} \label{Background}
Many existing games make use of a strong story focus, procedural narrative/quest generation, or some combination of both. Mainstream games at the time of this paper have yet to make use of procedural narrative generation but have used it in related elements such as quest generation. Procedural quest generation will dynamically assign players different tasks at different locations but usually does not feature a complex or detailed. The Elder Scrolls V: Skyrim \cite{skyrim} and Fallout 4 \cite{hernandez_2016} make extensive use of procedural quest generation which they call "Radiant Story" \cite{radiantAI}. The use of "Radiant Story" allows for potentially unlimited quests to be generated, with the downside of being extremely repetitive. Most Radiant Story quests involved unnamed generic NPCs, a simple goal (like kill all enemies at x location), and a limited set of dialogue lines. Many players have criticized certain radiant quests that were constantly triggered in Fallout 4.
